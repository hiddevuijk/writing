\section{Hierarchy}

Ungerleider and Mishkin \cite{mishkin1983object} put forward evidence for a two distinct cortical visual pathways, the ventral and dorsal pathway. The distinction is made on basis of functional processing; the ventral pathway is specialized in determining what an object is, and the dorsal pathway in determining the location of an object. Milner and Goodale \cite{milner2008two, goodale1992separate} also found the same two pathway, but in their view the ventral stream is specialized in perception and the dorsal stream in action. -- rewrite -- --read neuroscience book --

One of the fundamental organizational principles of the visual system in brain is the increase in the size of the receptive field along the a cortical visual pathway \cite{hubel1988eye,hubel1962receptive,lerner2001hierarchical} -- check ref --. In the primary visual cortex, an area early in the visual pathway, the receptive field is small. Further along the visual pathway neurons receive input from an increasing number of neurons, which have a smaller receptive field. This results in integration of stimuli over a larger spatial field, which enables brain functions such as size invariant object recognition \cite{kobatake1994neuronal}.Much research has been done on the spatial integration of information in the visual pathway \cite{deyoe1996mapping} -- put in some examples --. 


-- anatomical hierarchy study van essen, see refs in wandell visual field maps --


Inspired by the receptive field hierarchy it as been shown that a similar hierarchy exists for the temporal receptive windows (TRW) in the human brain, for both the visual as the auditory pathway. The  TRW is defined as the length of time before a response during which sensory information may affect that response -- copied --. Neurons in early sensory areas have a short TRW of the order of tens of milliseconds, which means that they integrate over a short period of time, and are, therefore, sensitive for rapid changes in the stimuli. Neurons in higher areas have a longer TRW, of the order of minutes, and retain information from a past stimuli, this enables these areas to process events that unfold over a certain time span\cite{hasson2008hierarchy, lerner2011topographic, honey2012slow}. A longer TRW is useful for perceptual and cognitive tasks such as segmenting ongoing activity in temporal parts \cite{zacks2001human,hasson2008hierarchy}, short-term memory \cite{hasson2015hierarchical, durstewitz2000neurocomputational}, inference of cause and effect \cite{fonlupt2003perception}, and processing language with a wide variety of time scales (e.g. complete narrative, paragraph, sentence, and words) \cite{xu2005language, lerner2011topographic}. Lerner et al. found that the TRWs of the brain areas rescale, up to a certain limit, when the incoming rate of information is changed \cite{lerner2014temporal}. Though the rate of incoming information can modulate the TRWs, the time scale gradient is an intrinsic property of neural circuits \cite{stephens2013place}.

-- theoretical/computational research --


